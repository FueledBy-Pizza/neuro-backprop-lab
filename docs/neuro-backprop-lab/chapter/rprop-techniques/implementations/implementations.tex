\begin{section}{Implementations}
    \par Recall that the main concern of the documentation is readability. Hence, pseudocode and actual code implementations may slightly differ, as the Python scripting language allows for significant performance improvements through the use of native structures. These differences clearly don't affect the functionality of the implementations.
    \par Each~\glsxtrshort{rprop} algorithm that will be described corresponds to a specialized\\\texttt{torch.optim.Optimizer.step()}\footnote{\textit{https://pytorch.org/docs/main/optim.html} (accessed 2025)} class method.\\
    \par An~\glsxtrshort{rprop} algorithm is intended to perform the following steps:
    \begin{enumerate}
        \item Compute the gradient of the error function with respect to the model weights.
        \begin{item}
            Update the step size based on a conditional logic of the current and previous gradient sign:
            \[
                \Delta_{ij}^{curr} =
                \begin{cases}
                    \min(\eta^{+} \cdot \Delta_{ij}^{prev}, \Delta_{\max}) & \text{if} {\frac{\partial E}{\partial w_{ij}}}^{curr} \cdot {\frac{\partial E}{\partial w_{ij}}}^{prev} > 0 \\[2ex]
                    \max(\eta^{-} \cdot \Delta_{ij}^{prev}, \Delta_{\min}) & \text{if} {\frac{\partial E}{\partial w_{ij}}}^{curr} \cdot {\frac{\partial E}{\partial w_{ij}}}^{prev} < 0 \\[2ex]
                    \Delta_{ij}^{prev} & \text{otherwise}
                \end{cases}
            \]
        \end{item}
        \item Update the step size direction using either weight-backtracking or the gradient sign.
        \item Update weights with the step size direction.
    \end{enumerate}
    Subsequently, each variant of the algorithm implements its own adaptation of this general process.
    \clearpage
    \begin{subsection}{Rprop-}
    \par This is Rprop-.
\end{subsection}
    \begin{subsection}{Rprop with Weight-Backtracking}
    \par This~\glsxtrshort{rprop} version, also said~\glsxtrshort{rprop}\textsuperscript{+}, updates the step size direction with the gradient sign when the gradient product is greater than or equal to zero. In the other case the algorithm performs a weight-backtracking, formally $\Delta w_{ij}^{curr} = -\Delta w_{ij}^{prev}$, then the current gradient is set to zero in order to activate the gradient-product case resulting in zero in the next iteration.
    \begin{subsection}{Rprop with Weight-Backtracking}
    \par This~\glsxtrshort{rprop} version, also said~\glsxtrshort{rprop}\textsuperscript{+}, updates the step size direction with the gradient sign when the gradient product is greater than or equal to zero. In the other case the algorithm performs a weight-backtracking, formally $\Delta w_{ij}^{curr} = -\Delta w_{ij}^{prev}$, then the current gradient is set to zero in order to activate the gradient-product case resulting in zero in the next iteration.
    \begin{subsection}{Rprop with Weight-Backtracking}
    \par This~\glsxtrshort{rprop} version, also said~\glsxtrshort{rprop}\textsuperscript{+}, updates the step size direction with the gradient sign when the gradient product is greater than or equal to zero. In the other case the algorithm performs a weight-backtracking, formally $\Delta w_{ij}^{curr} = -\Delta w_{ij}^{prev}$, then the current gradient is set to zero in order to activate the gradient-product case resulting in zero in the next iteration.
    \input{chapter/rprop-techniques/implementations/rpropplus/pseudocode/rpropplus}
    \clearpage
\end{subsection}
    \clearpage
\end{subsection}
    \clearpage
\end{subsection}
    \begin{subsection}{Improved Rprop with Weight-Backtracking}
    \par This~\glsxtrshort{rprop} version, also said~\glsxtrshort{irprop}\textsuperscript{+}, extends the one described in subsection~\ref{subsec:rpropplus}.
    \par The only modification concerns the gradient-product case resulting in less than zero: weight-backtracking is adopted only if the current error is greater than the previous error (this is what is meant by `improved'), otherwise the step size direction is set to zero.
    \begin{subsection}{Improved Rprop with Weight-Backtracking}
    \par This~\glsxtrshort{rprop} version, also said~\glsxtrshort{irprop}\textsuperscript{+}, extends the one described in subsection~\ref{subsec:rpropplus}.
    \par The only modification concerns the gradient-product case resulting in less than zero: weight-backtracking is adopted only if the current error is greater than the previous error (this is what is meant by `improved'), otherwise the step size direction is set to zero.
    \begin{subsection}{Improved Rprop with Weight-Backtracking}
    \par This~\glsxtrshort{rprop} version, also said~\glsxtrshort{irprop}\textsuperscript{+}, extends the one described in subsection~\ref{subsec:rpropplus}.
    \par The only modification concerns the gradient-product case resulting in less than zero: weight-backtracking is adopted only if the current error is greater than the previous error (this is what is meant by `improved'), otherwise the step size direction is set to zero.
    \input{chapter/rprop-techniques/implementations/irpropplus/pseudocode/irpropplus}
    \clearpage
\end{subsection}
    \clearpage
\end{subsection}
    \clearpage
\end{subsection}
    \begin{subsection}{Rprop with Weight-Backtracking by PyTorch}
    \par This is Rprop+ by PyTorch.\footnote{\textit{https://pytorch.org/docs/stable/generated/torch.optim.Rprop.html} (accessed 2025)}
\end{subsection}
\end{section}