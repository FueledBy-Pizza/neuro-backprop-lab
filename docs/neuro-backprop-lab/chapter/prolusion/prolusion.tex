\begin{chapter}{Prolusion}
    \begin{section}{Goal}
        \par This report provides a comprehensive overview of a Python project whose goal is to develop and compare different adaptive backpropagation techniques involved in a machine learning process, as~\glsxtrfull{rprop}.
		\par The project follows the ``Empirical evaluation of the improved Rprop learning algorithms'' article by Christian Igel and Michel Hüsken (2001).
    \end{section}
    \newpage
	\begin{section}{Software Stack}
		\begin{itemize}
			\item Python 3.9.6
			\item PyTorch 2.6.0
		\end{itemize}
		The project is equipped with a \texttt{requirements.txt} file which allows for seamless installation of dependencies, by executing \texttt{pip install -r requirements.txt}.
	\end{section}
	\newpage
	\begin{section}{Project Structure}
		\dirtree{%
			.1 neuro-backprop-lab/.
			.2 model/.
			.2 tester/.
			.3 tester.py.
			.3 {<}trained\_model{>}.pt.
			.2 trainer/.
			.3 irpropplus/.
			.3 rpropminus/.
			.3 rpropplus/.
			.3 trainer.py.
			.2 utils/.
			.2 test\_model.py.
			.2 train\_model.py.
		}
		\medskip
		\begin{itemize}
			\item \texttt{model} includes the neural network model architecture.
			\item \texttt{tester} handles the testing flow of the ready-to-use \texttt{<trained\_model>.pt}.
			\item \texttt{trainer} handles the examined backpropagation techniques and the training flow of the model, saving it as \texttt{<trained\_model>.pt}.
			\item \texttt{utils} offers utility functions designed to support the root project scripts.
		\end{itemize}
	\end{section}
\end{chapter}